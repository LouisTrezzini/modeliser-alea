\documentclass[12pt,a4paper]{article}

\usepackage[utf8]{inputenc}
\usepackage[french]{babel}
\usepackage[T1]{fontenc}
\usepackage[margin=1in]{geometry}
\usepackage{graphicx}
\usepackage{amsmath}
\usepackage{amsthm}
\usepackage{listings}
\usepackage{courier}
\usepackage{amsfonts}
\usepackage{stmaryrd}
\usepackage{diagbox}
\usepackage{mathtools}

\lstset{basicstyle=\footnotesize\ttfamily,breaklines=true}
\lstset{frame=single}
\lstset{language=Scilab}

\usepackage{float}

\setlength{\parindent}{0pt}
\setlength{\parskip}{0.5em}

\title{\textbf{TP2 “MODÉLISER L’ALÉA” \\ SIMULATION DE FILES D'ATTENTE}}
\author{Clément Riu - Louis Trezzini}
\date{29 mai 2017}

\newtheorem{lemme}{Lemme}

\begin{document}

\maketitle

\paragraph*{Question 1.} ~\\

\begin{figure}[H]
	\centering
	\includegraphics[width=0.80\textwidth]{Figure_1.eps}
	\caption{$\mathbf{X}_t$ en fonction du temps.}
\end{figure}
\begin{figure}[H]
	\centering
	\includegraphics[width=0.80\textwidth]{Figure_2.eps}
	\caption{$\mathbf{X}_t$ en fonction du temps.}
\end{figure}
\begin{figure}[H]
	\centering
	\includegraphics[width=0.80\textwidth]{Figure_3.eps}
	\caption{$\mathbf{X}_t$ en fonction du temps.}
\end{figure}

Comme on peut s'y attendre, lorsque l’espérance des départs est supérieure à l'espérance des arrivées, la quantité de personne en attente stagne dans de petites valeurs. Lorsque $\rho$ vaut 1, le résultat est irrégulier, avec des croissances et décroissances du nombre de personnes arrivant. Le nombre de personne en attente varie plus qu'avant mais reste faible. À l'inverse, lorsque $\rho > 1$, le nombre de personne en attente croit relativement vite.

\paragraph*{Question 2.}

Calculons la moyenne et l'écart type théorique : 

\begin{align*}
\mathbb{E}(X_t) &= \frac{\rho}{1 - \rho} \\
\text{var}(X_t) &= \frac{\rho}{(1 - \rho)^2}
\end{align*}

Dans notre cas on a $\rho = 0.5$: 

\begin{align*}
\mathbb{E}(X_t) &= 1 \\
\text{var}(X_t) &= 2
\end{align*}

On utilise deux méthodes différentes pour calculer la moyenne et la variance expérimentalement.

Première méthode, on se place en régime stationnaire (donc en $t$ grand) et on simule plusieurs fois la trajectoire. On applique alors la loi des grands nombre. Nos résultats sont alors :

\begin{align*}
\mathbb{E}(X_t) &= 1.641 \\
\text{var}(X_t) &= 1.672
\end{align*}

Deuxième méthode, on simule une longue trajectoire et par le théorème ergodique on applique les résultats du théorème 1 :

\begin{align*}
\mathbb{E}(X_t) &= 1.3792 \\
\text{var}(X_t) &= 1.6754
\end{align*}

Même si les ordres de grandeur sont bons, il y a une imprécision sur les résultats due à la discrétisation du temps continue.

\paragraph*{Question 3.}

En utilisant le théorème ergodique, on modélise la distribution de $\mathbf{X}_t$ via la distribution invariante. On a alors :

\begin{figure}[H]
	\centering
	\includegraphics[width=0.95\textwidth]{Figure_4.eps}
	\caption{Distribution en régime stationnaire.}
\end{figure}

\end{document}
